\documentclass[a4paper,oneside,12pt]{article}

\usepackage[utf8]{inputenc}
\usepackage[slovene]{babel}
\usepackage[T1]{fontenc}
\usepackage{textcomp}
\usepackage[
    paper=a4paper,
    top=2.5cm,
    bottom=2.5cm,
%    textheight=24cm,
    textwidth=15cm,
    ]{geometry}

\begin{document}
\begin{center}
  {\huge Iskanje optimalnega splošnega \\ kompozitnega sortirnega algoritma} \\[10mm]
 raziskovalec: Jure Slak \\[5mm]
 mentor: Klemen Bajec - Gimnazija Vič \\[3mm]
 somentor: Gašper Ažman - študent?  \\[5mm]
 ključne besede: sortirni algoritmi, urejanje s kompozitnim algoritmom, računalniku
 prilagodljivo urejanje, podatkom prilagodljivo urejanje 
 \\[10mm]
\end{center} 

{\centering {\large \textsc{povzetek}} \\[4mm]}
Namen raziskovalne naloge je poiskati optimalen sortirni algoritem.
V prvem delu raziskovalne naloge je opisana teorija urejanja, različni sortirni
algoritmi in njihovo pričakovano obnašanje na dejanskih podatkih. Postavljeni sta tudi dve 
hipotezi glede učinkovitosti sortirnih algoritmov. 
V nadaljevanju je opisana ideja, na
kateri temelji kompozitni sortirni algoritem in konfiguracija, s katero nadzorujemo
njegovo delovanje. Pomemben del je teoretična izpeljava
algoritma, ki s primerjanjem učinkovitosti posameznih sortirnih algoritmov poišče
optimalno konfiguracijo za kompozitni sortirni algoritem. Podan je tudi opis 
implementacije tega algoritma, ki mu sledi 
predstavitev rezultatov. Rezultati prikazujejo primerjavo učinkovitosti posameznih 
sortirnih algoritmov na različnih vrstah podatkov. 
Dobljene rezultate razložimo z vidika teoretičnih osnov in na podlagi naših ugotovitev 
ovrednotimo hipotezi. Rezultat raziskovalne naloge je učinkovit in izjemno prilagodljiv
kompozitni sortirni algoritem, ki se lahko prilagodi tako zmogljivosti strojne opreme
kot podatkom, ki jih ureja.

\end{document}
% vim: spell spelllang=sl
