\documentclass[a4paper,oneside,12pt]{article}

\usepackage[utf8]{inputenc}
\usepackage[slovene]{babel}
\usepackage[T1]{fontenc}
\usepackage{textcomp}
\usepackage[
    paper=a4paper,
    top=2.5cm,
    bottom=2.5cm,
%    textheight=24cm,
    textwidth=15cm,
    ]{geometry}

\begin{document}
\begin{center}
  {\huge Iskanje optimalnega splošnega \\ kompozitnega sortirnega algoritma} \\[10mm]
 raziskovalec: Jure Slak \\[5mm]
 mentor: Klemen Bajec - Gimnazija Vič \\[3mm]
 somentor: Gašper Ažman - študent?  \\[5mm]
 ključne besede:  sortirni algoritmi, urejanje s kompozitnim algoritmom\\[10mm]
\end{center} 

 {\centering {\large \textsc{povzetek}} \\[4mm]}
Raziskovalna naloga obravnava dva problema. Prvi je problem različne uspešnosti 
sortirnih algoritmov pri različnih lastnostih podatkov, ki se urejajo. Drugi problem pa je
vprašanje ali je pri urejanju učinkovitejši kompozitni algoritem ali pa je vsak posamezen
algoritem boljši. Prvi del raziskovalne naloge je posvečen teoriji sortiranja in različnim sortirnim
metodam, ki jih bomo primerjali med seboj in iz katerih bo sestavljen kompozitni
algoritem, ter njihovemu obnašanju v realnih primerih. V nadaljevanju je opisana ideja, na
kateri temelji kompozitni sortirni algoritem in konfiguracija s katero nadzorujemo
obnašanje kompozitnega sortirnega algoritma. Najpomembnejši del je teoretična izpeljava
algoritma, ki poišče optimalno konfiguracijo za kompozitni sortirni algoritem, pri
določenih podatkih. Opisani so tudi ostali algoritmi, s katerimi sem si pomagal pri
izvedbi eksperimenta. Sledi opis implementacije in eksperimentalnih pogojev, nato pa
predstavitev rezultatov, ki sem jih dobil za določene vrste podatkov.
Sledi razlaga rezultatov za vsak tip posebej, podprta s teoretičnimi osnovami, kjer sta
tudi rešena oba obravnavana problema. Rešitev je učinkovit in izjemno prilagodljiv
kompozitni sortirni algoritem, ki se lahko prilagodi tako računalniku kot podatkom, ki jih
ureja vendar je kljub temu enostaven za uporabo.

\end{document}
% vim: spell spelllang=sl
